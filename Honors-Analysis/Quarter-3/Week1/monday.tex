
\RequirePackage{luatex85}
\documentclass{amsart}
%-------Packages---------
\usepackage{fontspec}
\usepackage{amssymb,amsfonts}
\usepackage[all,arc]{xy}
\usepackage[shortlabels]{enumitem}
\usepackage{mathrsfs}
\usepackage{amsthm}
\usepackage{thmtools, thm-restate}
\usepackage[dvipsnames]{xcolor}
\usepackage{
nameref,%\nameref
hyperref,%\autoref
% n.b. \Autoref is defined by thmtools
cleveref,% \cref
% n.b. cleveref after! hyperref
}
\usepackage{listings}

\usepackage[margin=1.25in]{geometry}
\usepackage{tikz}

\usepackage{mathtools}
\DeclarePairedDelimiter{\ceil}{\lceil}{\rceil}
\DeclarePairedDelimiter{\floor}{\lfloor}{\rfloor}
\DeclarePairedDelimiter{\norm}{\lVert}{\rVert}
\DeclarePairedDelimiter{\inner}{\langle}{\rangle}
\newcommand{\abs}[1]{\left| #1\right|}
\newcommand{\parens}[1]{\left( #1\right)}
\newcommand{\bracks}[1]{\left[ #1\right]}
\newcommand{\braces}[1]{\left\{ #1\right\}}


%--------Theorem Environments--------
%theoremstyle{plain} --- default
\renewcommand*{\thesubsection}{\thesection.\ifnum\value{subsection}=0 M\else\ifnum\value{subsection}=1 W\else\ifnum\value{subsection}=2 F\else P\fi\fi\fi}
\declaretheorem[numberwithin=subsection]{theorem}
\declaretheorem[sibling=theorem]{corollary}
\declaretheorem[sibling=theorem]{proposition}
\declaretheorem[sibling=theorem]{lemma}
\declaretheorem[sibling=theorem]{conjecture}
\declaretheorem[sibling=theorem]{question}
\declaretheorem[sibling=theorem]{process}

\theoremstyle{definition}
\declaretheorem[sibling=theorem, style=definition]{definition}
\declaretheorem[sibling=theorem, style=definition]{definitions}
\declaretheorem[sibling=theorem, style=definition]{construction}
\declaretheorem[sibling=theorem, style=definition]{example}
\declaretheorem[sibling=theorem, style=definition]{examples}
\declaretheorem[sibling=theorem, style=definition]{notation}
\declaretheorem[sibling=theorem, style=definition]{notations}
\declaretheorem[sibling=theorem, style=definition]{convention}
\declaretheorem[sibling=theorem, style=definition]{addendum}
\declaretheorem[sibling=theorem, style=definition]{exercise}
\declaretheorem[sibling=theorem, style=definition]{consequence}
\declaretheorem[sibling=theorem, style=definition]{observation}
\declaretheorem[numbered=no, style=definition, name=Reminder Definition]{definition-prime}

\declaretheorem[sibling=theorem, style=remark]{remark}
\declaretheorem[sibling=theorem, style=remark]{axiom}
\declaretheorem[sibling=theorem, style=remark]{recall}
\declaretheorem[sibling=theorem, style=remark]{remarks}
\declaretheorem[sibling=theorem, style=remark]{warnings}
\declaretheorem[sibling=theorem, style=remark]{scholium}
\declaretheorem[sibling=theorem, style=remark]{solution}
\declaretheorem[sibling=theorem, style=remark]{notes}
\declaretheorem[numbered=no, style=remark]{problem}

\declaretheorem[numberwithin=section, shaded={bgcolor=Turquoise}]{homework}
\declaretheorem[sibling=homework, shaded={bgcolor=Lavender}]{challenge}
\declaretheorem[sibling=homework, name=Do, shaded={bgcolor=Green}]{do-this}
\declaretheorem[sibling=homework, name=Do!, shaded={bgcolor=Green}]{do-this!}
\declaretheorem[sibling=homework, name=Do*, shaded={bgcolor=Green}]{do-this*}
\declaretheorem[sibling=homework, name=Ah-Ha!, shaded={bgcolor=Green}]{ah-ha}

\declaretheorem[numberwithin=section]{history}
\declaretheorem[sibling=history]{architecture}

\usepackage{blindtext}

\makeatletter
\let\c@equation\c@theorem
\makeatother
\numberwithin{equation}{section}

\newcommand{\sm}{\setminus}
\newcommand{\s}[1]{\{#1\}}
\newcommand{\SET}[1]{\braces{#1}}
\newcommand{\n}{\not}
\newcommand{\nin}{\n\in}
\newcommand{\susbet}{\subset}

\newcommand{\IF}{\text{ if }}
\newcommand{\ST}{\text{ such that }}
\newcommand{\OR}{\text{ or }}
\newcommand{\AND}{\text{ and }}
\newcommand{\OTHERWISE}{\text{ otherwise }}
\newcommand{\WHERE}{\text{ where }}
\newcommand{\FORALL}{\text{ for all }}
\newcommand{\FOR}{\text{ for }}
\newcommand{\FORSOME}{\text{ for some }}
\newcommand{\THEN}{\text{ for some }}

\newcommand{\GCD}{\text{GCD}}
\newcommand{\ext}{\text{ext}}
\newcommand{\Span}{\text{span}}
\newcommand{\SO}{\text{SO}}
\newcommand{\diam}{\text{diam}}
\newcommand{\DET}{\text{det}}
\newcommand{\adj}{\text{adj}}
\newcommand{\GLNR}{GL_n(\R)}
\newcommand{\sgn}{\text{sgn}}
\newcommand{\Vol}{\text{Vol}}
\usepackage{bbm}
\newcommand{\ind}{\mathbbm{1}}
\newcommand{\len}{\text{len}}

\newcommand{\N}{\mathbb{N}}
\newcommand{\A}{\textbf{A}}
\newcommand{\R}{\mathbb{R}}
\newcommand{\Z}{\mathbb{Z}}
\newcommand{\I}{\textbf{I}}
\newcommand{\C}{\textbf{C}}
\newcommand{\F}{\textbf{F}}
\newcommand{\G}{\textbf{G}}
\newcommand{\HH}{\mathbb{H}}
\newcommand{\Q}{\mathbb{Q}}
\newcommand{\LL}{\textbf{L}}

\newcommand{\q}{\textbf{q}}
\newcommand{\g}{\mathscr{g}}

\newcommand{\va}{\textbf{a}}
\newcommand{\vb}{\textbf{b}}
\newcommand{\vc}{\textbf{c}}
\newcommand{\vd}{\textbf{d}}
\newcommand{\ve}{\textbf{e}}
\newcommand{\vf}{\textbf{f}}
\newcommand{\vs}{\textbf{s}}
\newcommand{\vt}{\textbf{t}}
\newcommand{\vu}{\textbf{u}}
\newcommand{\vv}{\textbf{v}}
\newcommand{\vw}{\textbf{w}}
\newcommand{\vx}{\textbf{x}}
\newcommand{\vy}{\textbf{y}}
\newcommand{\vz}{\textbf{z}}
\newcommand{\vze}{\textbf{0}}
\newcommand{\fin}{f^{-1}}
\newcommand{\pdv}[2]{\frac{\partial #1}{\partial #2}}

\newcommand{\limsupin}{\limsup_{n\to\infty}}
\newcommand{\limin}{\lim_{n\to\infty}}
\newcommand{\limft}[2]{\lim_{#1\to#2}}
\newcommand{\limtin}[1]{\lim_{#1\to\infty}}
\newcommand{\dlimin}{\underset{n\to\infty}\lim}
\newcommand{\dlimft}[2]{\underset{#1\to#2}\lim}
\newcommand{\dlimtin}[1]{\underset{#1\to\infty}\lim}
\newcommand*\diff{\mathop{}\!\mathrm{d}}
\newcommand{\sumin}[1]{\sum{#1=1}^\infty}

\renewcommand{\phi}{\varphi}
\newcommand{\tilR}{\overset{\sim}{R}}

\newcommand{\onto}{\twoheadrightarrow}

\usepackage{changepage}
\usepackage{luacode}
\usepackage{pgfplots}
\bibliographystyle{plain}

\setcounter{section}{1}
\setcounter{subsection}{0}
\title{Day 1M: Lebesgue Measure}
\date{March 26 2018}
\author{Lecturer: Charlie Smart}

\begin{document}

  \maketitle

  \section*{Administrivia}

  \begin{enumerate}
    \item[Teacher] Charlie Smart (smart@math.uchicago.edu)
    \item[TA] Peter Morfe (pmorfe@math.uchicago.edu)
  \end{enumerate}

  \section*{Measure Theory}
  Measure theory is the abstract or algebraic theory of size. We will be starting with arbitrary sets $A \subset \R^d$ using the \emph{Lebesgue Measure}. To do so we will be slightly generalizing the Riemann integral. The volume of set can be seen as the sum of volumes of the rectangles which most efficiently cover the set.

  \begin{definition}

    A \emph{rectangle} is a product of intervals
    \[R = [a_1, b_1] \times [a_2, b_2] \times \cdots \times [a_d, b_d] \subseteq \R^d\]
    which has volume
    \[|R| = \prod_{i=1}^d (b_i - a_i).\]

  \end{definition}

  \begin{definition}

    If $A \subset \R^d$, then the \emph{Lebesgue outer measure} is
    \[\mu(A) = \inf \SET{\sum_{k \geq 1}|R_k| : \s{R_k}_{n\in\N} \text{ are rectangles s.t. } A \subset \bigcup_{k\geq1}R_k}.\]

    This is like the Riemann integral in that it is a sum of squares where the indicator function is one somewhere, but it allows infinitely many rectangles as oposed to finitely many.

  \end{definition}

  \begin{example}

    As an example to show why we need infinitely many rectangles, we will show that $\mu(\Q \cap [-1, 1]) = 0$.

  \end{example}

    \begin{proof}

      Let $\s{r_1, r_2, \ldots} = \Q \cap [-1, 1]$. Fix $\epsilon > 0$. Then, we note that
      \[\Q \cap [-1, 1] \subseteq \bigcup_{k \geq 1}[r_k - \epsilon 2^{-k}, r_k + \epsilon 2^{-k}].\]
      Furthermore,
      \[\sum_{k\geq1}|[r_k - \epsilon 2^{-k}, r_k + \epsilon 2^{-k}]| = \sum_{k\geq1}\epsilon 2^{1-k} = 2\epsilon.\]
      Thus, $\mu(A) \leq \epsilon$ for all $\epsilon > 0$ and therefore is $0.$

    \end{proof}

    \begin{do-this}

      Show that if $\Q \cap [-1, 1] \subset I_1 \cup I_2 \cup \cdots \cup I_n$ then
      \[|I_1| + \cdots + |I_n| \geq 2.\]
      And therefore, to measure $\Q \cap [-1, 1]$ we need infinitely many rectangles.

    \end{do-this}

  The question is then whether or not $\mu$ is a reasonable measure.

  Some thing which contains another thing should be bigger:
  \begin{theorem}

    If $A \subset B \subset \R^d$ then $\mu(A) \leq \mu(B)$.

  \end{theorem}

    \begin{proof}

      Suppose $\s{R_k}$ covers $B$. Then it will also cover $A.$ So,
      \[\SET{\sum_{k \geq 1}|R_k| : \s{R_k}_{n\in\N} \text{ are rectangles s.t. } A \subset \bigcup_{k\geq1}R_k} \subset \SET{\sum_{k \geq 1}|R_k| : \s{R_k}_{n\in\N} \text{ are rectangles s.t. } B \subset \bigcup_{k\geq1}R_k}.\]
      Therefore, by definition of infimum, $\mu(A) \leq \mu(B)$.

    \end{proof}

  Ideally, the volume of a bunch of things fused together will be smaller than the sum of its composite parts.
  \begin{theorem}

    Suppose $\s{A_k}_{k\in\N}$ are all subsets of $\R^d$. Then,
    \[\mu(\bigcup_{k \geq 1}A_k) \leq \sum_{k \geq 1}\mu(A_k).\]

  \end{theorem}

    \begin{proof}

      Suppose $\sum_{k \geq 1}\mu(A_k) = \infty$. Then this is clearly true. Now we suppose that sum is finite.

      Fix $\epsilon > 0$. For each $k$, choose a cover $\s{R^k_j}_{j\in\N}$ of $A_k$ so that
      \[\sum_{j \geq 1}|R_j^k| \leq \mu(A_k) + \epsilon 2^{-k}.\]
      Such a cover exists by the definition of infimum. Now,
      \begin{align*}
        \bigcup_{k \geq 1}A_k &\subset \bigcup_{k \geq 1} \bigcup_{j \geq 1}R_j^k &&\text{so by definition,}\\
        \mu(\bigcup_{k \geq 1}A_k) &\leq \sum_{k \geq 1}\sum_{j \geq 1} |R_j^k|\\
        &\leq \sum_{k \geq 1}[\mu(A_k) + \epsilon 2^{-k}]\\
        &= \bracks{\sum_{k \geq 1}\mu(A_k)} + \epsilon.
      \end{align*}

      As this holds for all $\epsilon > 0,$
      \[\mu(\bigcup_{k \geq 1}A_k) \leq \sum_{k \geq 1}\mu(A_k)\]

    \end{proof}

  \begin{definition-prime}

    The \emph{interior} of a set $A \subset \R^d$ is given by $A°$ and is the unique open set contained in $A$ which contains all other open subsets of $A$.

  \end{definition-prime}

  \begin{definition-prime}

    The \emph{indicator function} of a set $A \subset \R^d$ is the function $\ind_A:\R^d \to \R$ defined as
    \[\ind_A(x) = \begin{cases}1 & \IF x \in A\\ 0 & \IF x \nin A.\end{cases}\]

  \end{definition-prime}

  \begin{recall}

    If $E \subset \R^d$ then \[\Vol(E) = \int_\R \ind_E(x) \diff x.\]

  \end{recall}

  Hopefully, the thing we use to measure other things will have the same measure as its volume.
  \begin{theorem}

    Suppose $R$ is a rectangle. Then, $\mu(R) = |R|.$

  \end{theorem}

    \begin{proof}

      To show this, it suffices to show for a collection of rectangles $\s{R_k}$ which covers $R,$ we have that $|R| \leq \sum_{k \geq 1}|R_k|$.

      First, for all $k$, we dilate $R_k$ to $\tilR_k$ so that $R_k \subset \tilR_k°$ and
      \[|\tilR_k| \leq |R_k| + \epsilon 2^{-k}.\]
      Then, $R \subseteq \bigcup_{k\geq1}R_k \subset \bigcup_{k\geq1}\tilR_k°.$ However, $R$ is compact, so we can choose a finite subcover $\s{\tilR_1, \ldots, \tilR_n}$ for some finite $n$. Thus, $R \subset \bigcup_{k=1}^n\tilR_k.$

      \begin{align*}
        \sum_{k=1}^n|\tilR_k| &\leq \sum_{k=1}^n(|R_k| + \epsilon 2^{-k})\\
        &\leq \parens{\sum_{k=1}^n|R_k|} + \epsilon\\
        &\leq \parens{\sum_{k\geq1}|R_k|} + \epsilon.
      \end{align*}

      (1) Therefore, it is enough to show that if
      \[R \subset \bigcup_{k=1}^n R_k \THEN |R| \leq \sum_{k=1}^n|R_k|.\]

      We claim that $\ind_R \leq \sum_{k=1}^n\ind_{R_k}$. Suppose $x \in R$. Then, there exists some $j \in [n]$ such that
      $x \in R_j$. Thus, \[\ind_R(x) = 1 = \ind_{R_j}(x) \leq \sum_{k=1}^n\ind_{R_k}(x).\]
      Suppose $x \nin R$. Then, $\ind_R(x) = 0 \leq \sum_{k=1}^n\ind_{R_k}(x)$ as those functions are bounded below by $0$. Therefore we have our claim. This implies that

      \begin{align*}
        |R| = \int_{\R^d}\ind_R(x)\diff x &\leq \int_{\R^d}\sum_{k=1}^n\ind_{R_k}(x)\diff x\\
        &= \sum_{k=1}^n \int_{\R^d}\ind_{R_k}(x) \diff x &&\text{by linearity.}\\
        &= \sum_{k=1}^n|R_k|.
      \end{align*}

      Therefore we have shown (1) and the theorem holds.

    \end{proof}

  \begin{remark}

    The elementary proof in the text essentially rebuilds the Riemann integral from scratch. To fully prove this rigorously, one would need to use induction on total number of elements.

  \end{remark}

\end{document}
